\subsection{Geometry}

See
\url{https://github.com/brettviren/gegede/blob/master/doc/others.org}
for notes on a survey of some players in this problem space (CAD,
LCDD, Unified Solids and GeGeDe as described in other docs in that
directory).  Components of this problem space:

\begin{description}
\item[Authoring] of geometry descriptions
  \begin{itemize}
  \item Sole source authoring, provide single point of definitive description
  \item Simple configuration language for end-users to tweak knobs
  \item Powerful language for expert developers to describe geometry
  \item General programming language vs. DSL (eg, direct authoring of GDML/AGDD or other XML)
  \item Describe both ideal (algorithmic) geometry and one with as-built or alignment offsets from the ideal.
  \item Allow for multiple developers to work together on different parts
  \item Be level in prerequisites (no massive training, no large expense, no proprietary applications)
  \end{itemize}
\item[Version and provenance] of geometries
  \begin{itemize}
  \item Allow for evolving and competing versions to exist
  \item Indelibly ``mark'' a version such that no later modifications can be added
  \item Track version through its use by the experiment software
  \end{itemize}
\item[File representation and conversion] from the description
  \begin{itemize}
  \item Must support the ``big names'' (Geant4 and ROOT and ???)
  \item Support for other consumers (AGDD/GraXML, OpenInventor, ???)
  \item Sole source means no input requirements
  \end{itemize}
\item[Visualizing] of the geometry
  \begin{itemize}
  \item ``Drill down'' by removing outer volumes to see inner ones
  \item Zoom, pan, rotate
  \item Visualize obvious overlaps, misplacements
  \item Produce PR quality pictures, raytrace
  \item Event visualization has overlap but not considered in scope (?)
  \end{itemize}
\item[Validating] the geometry
  \begin{itemize}
  \item Detailed overlap detection and reporting in a way that eases fixing
  \item Correctness (how?)
  \end{itemize}
\item[Applying] the geometry 
  \begin{itemize}
  \item Performance in Geant4 and ROOT (Unified Solids effort)
  \item Optimizing tracking algorithms
  \item Feedback from tracking to find optimization in the geometry organization 
  \end{itemize}
\end{description}
