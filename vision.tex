\section{Vision Statement}

The working group members choose to take this opportunity to describe
what we would like to see materialize to improve beneficial
cross-experiment software development and use.

\subsection{HEP Software Foundation}

The HEP Software Foundation\cite{hsfwhitepaper} (HSF) is forming at
the time of the writing of this report.  It shares many of the goals
of the working group. The eventual FCE should work with HSF in ways
that can leverage its unique position.  

\subsection{Community}

One or more nuclei, around which the HEP software community can
consolidate to facilitate better cross-experiment software development
and use are needed.  For a community nucleus to be beneficial it must be created
with understanding and sensitivity to how the members of the community
naturally work.  Any Internet-based forum that lacks this will simply
be left unused.  Specific aspects of a community nucleus may include:

\begin{itemize}
\item Email lists specific to the various areas of software.  They are to be
  newly created and hosted if missing or advertised if existing.
\item A ``market'' where experiments can go to advertise their needs,
  solicit ideas from a broad swath of the community and form 
  collaborative efforts to satisfy these needs.
\item An ``incubator'' where novel development or improvement and
  generalization of existing software can be discussed and directed
  through community involvement and when possible contributed to by
  available community members.  A component of this incubator would
  include developing funding proposals.  (See also section \label{sec:support}.)
\item A ``knowledge-base'' filled with collaboratively produced content
  that includes items such as:
  \begin{itemize}
  \item summary information on individual software libraries and tools
    which are considered useful by and for the community.
  \item experiments, projects and other organizations and associations with the software they use.
  \item contact information on community individuals and the software
    and experiments they are involved in.
  \item an indexed archive of software documentation, publications,
    ``how to'' guides maintained directly in the knowledge-base or 
    external links to existing ones.
  \end{itemize}
\end{itemize}

\noindent These online resources should be open for viewing by the world and
indexed by Internet search engines to facilitate information
discovery.

The community should receive reports from the leaders of the effort to
improve the benefit of cross-experiment software as part of existing
HEP software and computing conferences.


\subsection{Support}
\label{sec:support}

We would like to see a method where novel effort beneficial to
multiple experiments can be funded.  The desired nature of such
funding is described here.

Funding proposals may come out of the ``incubator'' described above but need not.
The effort may be toward making an existing software more generally
beneficial to multiple experiments
or toward providing new development or reimplementation that is 
considered beneficial for and by the community.  This may include factoring
general parts from experiment-specific libs/tools, adoption of one
experiment's software by another or a ground up novel design and
implementation which provides a more general instance of existing
experiment-specific software.

For such proposed effort, a balance must be struck between addressing
the needs of one (or more) specific experiment and having the
development of a proper, general solution not be rushed by those
immediate needs.  Proposals may be born from the needs of a specific
experiment but should be judged by how useful the proposed solution
will be to experiments in the future or the larger existing community.
The effort should not be negatively judged solely due to not being
concluded in time to be used by the originating experiment.

Such proposals should be specific to the development of some software
deliverable.  They should include expected mechanisms to provide long
term support and improvement but funding for any ongoing effort should
not be part of such proposals.  Proposals may be for effort to
accomplish some specific novel deliverable or one which augments or
improves existing software.

Software resulting from such supported effort must be made available
to the HEP community on terms consistent with an  established
Free Software or Open Source license such as the GNU GPL or BSD licenses.

