\section{Vision Statement}

The working group members chooses to take this opportunity to describe
what we would like to see materialize to improve the beneficial
cross-experiment software development and use.

\subsection{HEP Software Foundation}

The HEP Software Foundation\cite{hsfwhitepaper} (HSF) is forming at
the time of the writing of this report.  It shares many of the goals
of the working group.  The eventual FCE should work with HSF in ways
that it can leverage its unique position.  

\subsection{Community}

One or more nuclei, around which the HEP software community can
consolidate to facilitate better cross-experiment software development
and use.  For a community nucleus to be beneficial it must be created
with understanding and sensitivity to how the members of the community
naturally work.  Any Internet-based forum that lacks this will simply
be left unused.  Specific aspects of a community nucleus may include:

\begin{itemize}
\item Email lists specific to the various areas of software to be
  newly created and hosted if missing or advertised if existing.
\item A ``market'' where experiments can go to advertise their needs,
  solicit ideas from a broad swath of the community and form
  collaborative efforts to satisfy these needs.
\item An ``incubator'' where novel development or improvement and
  generalization of existing software can be discussed and directed
  through community involvement and when possible contributed to by
  available community members.  A component of this incubator would
  include developing funding proposals.  Proposals should be made
  independent from any tied to a particular experiment.
\item A ``knowledge-base'' filled with collaboratively produced content
  that includes items such as:
  \begin{itemize}
  \item summary information on individual software libraries and tools
    which are considered useful by and for the community.
  \item experiments, projects and other organizations and the software they use.
  \item contact information on community individuals and the software
    and experiments they are involved in.
  \item an indexed archive of software documentation, publications,
    ``how to'' guides or links to these
  \end{itemize}
\end{itemize}

\noindent These online resources should be open for viewing by the world and
indexed by Internet search engines to facilitate information
discovery.

The community should receive reports from the leaders of the effort to
improve the benefit of cross-experiment software as part of existing
HEP software and computing conferences.


\subsection{Support}

We would like to see a method where novel effort can be funded.
Funding proposals may come out of the ``incubator'' described above.
The effort should be toward making an existing software more generally
beneficial to multiple experiments.  This may include factoring
general parts from experiment-specific ones, adopting of one
experiment's software by another or a ground up novel design and
implementation which provides a more general instance of existing
experiment-specific software.

To the extent possible, such effort should not be constrained to solve
any pressing emergency or immediate problems but be allowed to develop
a design and implementation which is proper for solving the problem in
a general and sustainable manner.

Such proposals should be for a limited scope and time and not to set
up to create new long standing effort.  Proposals may be for partial
support to augment existing effort.

Software resulting from such supported effort must be made available
to the HEP community on terms consistent with a common and established
Free Software or Open Source license.

\subsection{Advocacy}

\fixme{What to put here?}

