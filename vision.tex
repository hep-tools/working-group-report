\section{Vision Statement}

The working group members chooses to take this opportunity to describe
what we would like to see materialize to improve the beneficial
cross-experiment software development and use.

\subsection{Community}

One or more nuclei, around which the HEP software community can
consolidate to facilitate better cross-experiment software development
and use.  For a community nucleus to be beneficial it must be created
with understanding and sensitivity to how the members of the community
naturally work.  Any Internet-based forum that lacks this will simply
be left unused.  Specific aspects of a community nucleus may include:

\begin{itemize}
\item Email lists specific to the various areas of software to be
  newly created if missing or advertised if existing.
\item A ``market'' where experiments can go to advertise their needs,
  solicit ideas from a broad swath of the community and form
  collaborative efforts to satisfy these needs.
\item A knowledge-base filled with collaboratively produced content
  that includes items such as:
  \begin{itemize}
  \item summary information on individual software libraries and tools
    which are considered useful by and for the community.
  \item experiments, projects and other organizations and the software they use.
  \item contact information on community individuals and the software
    and experiments they are involved in.
  \item an indexed archive of software documentation, publications,
    ``how to'' guides or links to these
  \end{itemize}
\end{itemize}

\noindent These online resources should be open for viewing by the world and
indexed by Internet search engines to facilitate information
discovery.

The community should receive reports from the leaders of the effort to
improve the benefit of cross-experiment software as part of existing
HEP software and computing conferences.


\subsection{Support}

We would like to see a method where explicitly funding can be given to
support developers to make promising software that was developed in a
limited context to be improved so that it may benefit a wider,
cross-experiment one.  Likewise, we would like to see funding made
available to support the additional up-front effort needed by new
experiments to adopt existing software.

\subsection{Advocacy}

What to put here?

