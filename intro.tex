\section{Introduction}

%% This is largely ripped from my (bv) invitation letter from Rob/Salman
The High-Energy Physics Forum for Computational Excellence (HEP-FCE)
was formed by the Department of Energy as a follow-up to a recent
report from the Topical Panel on Computing\cite{topicalpanel} and the
associated P5 recommendation\cite{p5}.
%
It is a  pilot project distributed across the DOE Labs.  
%
During this initial incubation period the Forum is to develop a plan
for a robust, long-term organization structure and a functioning web
presence for forum activities and outreach, and a study of hardware
and software needs across the HEP program.

Three FCE Working Groups have been constituted as a limited-duration effort 
to list and prioritize computational activities across the three HEP frontiers.
This report summarizes the deliberations of the HEP-FCE working group which 
is focused on software libraries and tools. The charge to the working group 
includes topics such as:

\begin{itemize}
\item Code management utilities
\item Build/release/scripting/testing tools
\item Documentation tools
\item Graphics packages
\item General purpose libraries (I/O, statistical analysis, linear algebra)
\item  Data management and transfer tools
\item  Workflow and Workload management
\end{itemize}

The other two working groups focus on Systems (computing, data and networking)
and Applications (the software itself and its distribution).   
%
Even after narrowing to just software libraries and tools, the breadth and
depth of work relevant to HEP is far too extensive to cover in this
document.  Instead of attempting to be comprehensive the working group
has considered only a sampling, hopefully representative, of the
possible projects and areas.  Omissions are not intended to be
interpreted as positive or negative reflections on those projects or
areas.

In the following sections we give this working group's ``vision'' for
aspects and qualities we wish to see in a future HEP-FCE.  We then
give a prioritized list of technical activities with suggested scoping
and deliverables that can be expected to provide cross-experiment
benefits.  The remaining bulk of the report gives a technical survey of some
specific ``areas of opportunity'' for cross-experiment benefit in the
realm of software libs/tools.  This survey serves as support for the
vision and prioritized list.  For each area we describe the ways that
cross-experiment benefit is achieved today, as well as describe known
failings or pitfalls where such benefit has failed to be achieved and
which should be avoided in the future.  For both cases, we try to give
concrete examples.  Each area then ends with an examination of what
opportunities exist for improvements in that particular area.


