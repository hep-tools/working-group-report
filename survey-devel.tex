\section{Software Development}

\subsubsection{Editing notes}

\textit{(authors: Brett, Peter)}

\textit{This is not a real section and will be deleted}

\begin{itemize}
\item Things to cover
  \begin{itemize}
  \item Development tools such as profilers, static analysis, debuggers
  \end{itemize}
\item Inconsistent, non-existent or simply crappy/cobbled software release build tools
  \begin{itemize}
  \item lack of knowledge, expertise and/or adoption of prevailing methods
  \item solving immediate, local problems instead of addressing root issues and larger context
  \end{itemize}
\item Ditto for development environment tools
  \begin{itemize}
  \item differences between dev env requirements for HEP compared to the Free Software world
  \item users are developers, need multiple versions installed at once, non-OS/system/root installation
  \item dependency management: development must be cognizant of changes by others in parts lower in the stack as well as how ones own changes affect parts above them in the stack
  \end{itemize}
\item Software release policies, mechanisms.  Automation.  Braindead policies.  Capricious changes in policy.  Documentation, planning.
\item Continuous Integration and unit tests
\item End-user environment management and package aggregation (environment variable based (UPS/EM) vs. file system based (single root or the Nix approach)
\item Repository and collections of repository.  Private vs. public.  Monolithic vs fine-grained.  Psychology of developing in a parochial environment.  \textit{Eg, committing to Fermilab Redmine or Daya Bay's  isolated, private SVN repository tends to make myopic in my design.  When I know commits are going to GitHub I take a moment when I make decisions to think, ``how can this be useful in a wider context''.}
\end{itemize}

\subsection{Description}
\subsection{What works, what doesn't}
\subsection{Examples}
\subsection{Opportunity for improvement}
