\section{Data Management}
\label{data}

\fixme{(authors: Anders, Brian, Maxim, Mike, Simon)}

\subsection{Definition}
Data Management is any content neutral interaction with the data and includes Data Storage/Archival, Access, 
Distribution and Curation. It includes technical solutions, procedures and policies and deals with the full life 
cycle of the data. 



\subsubsection{Editing notes}

\fixme{This is not a real section and will be deleted}

\begin{itemize}
\item Define data management and it's major elements (one suggestion is below)
\item reference that it is related to workflow management
\item List ways that DM can tend to be particularly parochial and how in some cases it must be specifically tailored to the computing facility and/or experiment.
\item List what elements are general.
\item Take care not to overlap with the systems group, but to point out where there are areas of overlap.
\end{itemize}
Possible partitioning of DM into smaller parts:
\begin{description}
\item[Distribution] issues:
  \begin{itemize}
  \item authentication/authorization
  \item caching / purging
  \item side hardware and software requirements
  \item on-demand vs. scheduled
  \end{itemize}
\item[Metadata] issues:
  \begin{itemize}
  \item What job produced what file and with what input parameters/data
  \item Where is my file? Data catalogs ...
  \item Fileset definitions
  \item File popularity to drive cache purges
  \item Important analysis summary info
  \item Locating files/events/objects
  \item Everything as metadata
  \item Provenance tracking
  \item Namespace issues
  \end{itemize}
\end{description}



\subsection{Description}
\subsection{What works, what doesn't}
\subsection{Examples}
\subsection{Opportunity for improvement}
