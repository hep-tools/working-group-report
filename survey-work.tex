\section{Workflow and Workload Management}

\subsubsection{Editing notes}

\textbf{(authors: Anders, Maxim, Mike, Peter)}

\textit{This is not a real section and will be deleted}

Note: I mark my comments as -mxp- for brevity - Maxim

\begin{itemize}
\item -mxp- there is workflow mgt and workload mgt. While they overlap at times, and also include varying degree of data management functionality, it is best to make this distinction (also based on existing literature). For that reason, I changed the title of the section.
\item -mxp- DIRAC, PanDA, Phedex.


\end{itemize}
Possible sub-categories
\begin{description}
\item[Processing] issues:

  \begin{itemize}
  \item Submission tools:
  \item Running multiple instances of a job differing only by parameters or input files
  \item Running jobs in parallel on local batch or wider Grid
  \item Workflow engines
  \end{itemize}
\item[Monitoring] issues:
  \begin{itemize}
  \item Overall health, throughput, usage, abuse
  \item Locating and understanding failures
  \item Determining systemic problems
  \item Drilling down to details at the level of individual jobs
  \end{itemize}
\item[Recovery] issues:
  \begin{itemize}
  \item rerun failed jobs
  \item clean up droppings
  \item fix systemic problems
  \end{itemize}
\end{description}

\subsection{WMS}
\subsection{Description}
According to a common definition, Grid computing is the collection of computer resources from multiple locations to reach a common goal. Cloud computing is essentially evolution of the same concept, with implied higher degree of computing resources and data storage abstraction, connectivity and transparency of access. In the following, we won't distinguish between these two concepts unless absolutely necessary.



\subsection{test}
\subsubsection{What works, what doesn't}
\subsubsection{Examples}
\subsubsection{Opportunity for improvement}
