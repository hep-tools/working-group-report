\newpage
\section{Workflow and Workload Management}
\subsection{The Challenge of the Three Domains}
In the past three decades, technological revolution in industry has enabled and was paralleled by the growing complexity in the field of scientific computing, where 
more and more sophisicated methods of data processing and analysis were constantly introduced, oftentimes at a dramatically increased scale.
Processing power and storage were becoming increasingly decentralized, leading to the need to manage these distributed resources in an optimal manner.
On the other hand, increasing sophistication of scientific workflows created the need to support these workflows in the new distributed computing medium.
Rapid evolution of the field such as HEP and time pressures to deliver in this competetive environment led to design and implementation of complete (to varying degrees)
and successful solutions to satisfy the needs of specific  research projects. In general, this had two consequences:

\begin{itemize}
\item Integrated and oftentimes -- not always -- project-specific design of workflow and workload management (see~\ref{workflow_workload} for definitions).
\item Tight coupling of workflow and workload management to data handling components.
\end{itemize}

We observe that there are essentially \textbf{three interconnected domains} involved in this subject: Workflow Management, Workload Management,
and Data Management. In many systems (cf. Pegasus~\cite{pegasus}) some of these domains can be ``fused'' (e.g. Workflow+Workload).
In the following, we  bring together a few standard definitions and real life examples to help clarify
relationships among these domains and in doing so form the basis for possible HEP-FCE recommendations. Our goal will be twofold:
\begin{itemize}
\item to identify the features and design considerations proven to be successful and which can serve as guidance going forward.
\item to identify common design and implementation elements and to develop understanding of how to enhance \textit{reusability} of existing and future systems of this kind.
\end{itemize}

\subsection{Description}

\subsubsection{Grid and Cloud Computing}
According to a common definition, \textbf{Grid Computing} is the collection of computer resources from multiple locations to reach a common goal. Oftentimes additional characteristics added to this include decentralized
administration and management and adherence to open standards. It was formulated as a concept in early 1990s, and motivated by the fact that computational tasks handled by large research projects reached
the limits of scalability of most individual computing sites. On the other hand, due to variations in demand, some resources were underutilized at times. There is therefore a benefit in implementing a federation of
computing resources, whereby large spikes in demand can be hadled by federated sites, while ``backfilling'' the available capacity with lower priority tasks submitted by a larger community of users.
Technologies developed in the framework of Grid Computing (such as a few reliable and popular types of \textit{Grid Middleware}) became a major enabling factor for many scientific collaborations including
nuclear and high-energy physics.

\textbf{Cloud Computing} is essentially an evolution of the Grid Computing concept, with implied higher degree of computing resources and data storage abstraction, connectivity and transparency of access.
In addition, Cloud Computing is characterized by widespread adoption of \textit{virtualization} - which is also used in the traditional Grid environment but at a smaller scale.
At the time of writing, Cloud prominently figures in the context of commercial services available on the network, whereby computing resources can be ``rented'' for a fee in the form of a Virtual Machine allocated
to the user, or a  number of nodes in the Cloud can be dynamically assigned to perform a neccesary computational task - as a transient, ephemeral resource. Such dynamic, on-demand characteristic
of the Cloud led to it being described as an ``elastic'' resource (cf. ``Amazon Elastic Compute Cloud'' - the EC2).

Regardless of the obvious differentiation of Cloud computing (due to its characteristics as \textit{a utility computing platform} and pervasive reliance on virtualization), many of its fundamental concepts and challenges
are common with its predesessor, Grid Computing. In the following, we won't distinguish between the two unless  necessary. In fact, the boundary is blurred even further by enhancing Grid middleware with tools
based on virtualization and Cloud API which essentially extend Grid resources with on-demand, elastic Cloud capability~\cite{star_acat11}.
The two are often seen as ``complementary technologies that will coexist at different levels of resource abstraction'' \cite{atlas_chep13}.


%
% There is also the need for seemless lightweight unit testing, regression testing and module development on small local systems outside of large grid or cloud environments.\\
% -mxp- Brian - good point but it obfuscates the simple statement re: cloud. I just wanted this t obe minimalistic - we will definitely put this statement in the document, just in a different loication.

\subsubsection{From the Grid to Workload Management}
\label{from_grid_to_workload}
Utilization of the Grid sites via appropriate middelware does establish a degree of resource federation, but it leaves it up to the user to manage data movement and job submission to multiple sites,
track job status, handle failures and error conditions, aggregate bookkeeping information and perform many other tasks. In absense of automation, this does not scale very well and limits the efficacy
of the overall system.

It was therefore inevitable that soon after the advent of reliable Grid middleware multiple physics experiments and other projects started developing and deploying \textbf{Workload Management Systems (WMS)}.
 According to one definition,
\textit{``the purpose of the Workload Manager Service (WMS) is to accept requests for job submission and management coming from its clients and take the appropriate actions to satisfy them''}~\cite{egee_user_guide}.
Thus, one of the principal functions of a Workload Management System can be described as ``brokerage'', in the sense that it matches resource requests to the actual distributed resources
on multiple sites. This matching process can include a variety of factors such as access rules for users and groups, priorities set in the system, or even data locality - which is in fact an important and interesting part of this process~\cite{panda_chep10}.

In practice, despite differing approaches and features, most existing WMS appear to share certain primary goals, and provide solutions to achieve these (to a varying degree). Examples:

\begin{itemize}

\item Insulation of the user from the complex and potentially heterogeneous environment of the Grid, and shielding the user from common failure modes of the Grid infrastructure.

\item Rationalization, bookkeeping and automation of software provisioning - for example, distribution of binaries and configuration files to multiple sites.

\item Facilitation of basic monitoring functions, e.g. providing efficient ways to determine the status of jobs being submitted and executed.

\item Prioritization and load balancing across the computing sites.

\item Implementation of interfaces to external data management systems, or actual data movement and monitoring functionality built into certain components of the WMS.

\end{itemize}

We shall present examples of existing WMS in one of the following sections.

\subsubsection{Workflow vs Workload}
\label{workflow_workload}
A \textit{scientific workflow} system is a specialized case of a \textbf{workflow management system}, in which computations and/or transformations and exchange of data are performed according to a defined set of rules
in order to achieve an overall goal ~\cite{grid_workflow_taxonomy,grid_workflow_fit,pegasus}. In the context of this document, this process involves execution on distributed resources. Since the process is
typically largely (or completely) automated, it is often described as ``orchestration'' of execution of multiple interdependent tasks. Workflow systems are sometimes described using the concepts of a \textit{control flow},
which refers to the logic of execution, and \textit{data flow}, which concerns itself with the logic and rules of transmitting data. There are various patterns identified in both control and data flow~\cite{workflow_patterns}.
Complete discussion of this subject is beyond the scope of this document.

A simple and rather typical example of a workflow is often found in Monte Carlo studies performed in High Energy Physics and related fields, where there is a chain of processing steps similar to the pattern below:
\\
\\
\textit{Event  Generation $\Longrightarrow$ Simulation $\Longrightarrow$ Digitization $\Longrightarrow$ Reconstruction}
\\
\\
Patterns like this one may also include optional additional steps (implied or made explicit in the logic of the workflow) such as merging of units of data (e.g.files) for more efficient storage and transmission.
Even the most trivial cases of processing, with one step only, may involve multiple files in input and/or output streams, which creates the need to manage this as a workflow. Oftentimes, however,
workflows that need to be created by researchers are quite complex. At extreme scale, understanding the behavior of scientific workflows becomes a challenge and an object of studies in its own right~\cite{panorama}.

Many (but not all) workflows important in the context of this document can be modeled as
\textbf{Directed Acyclic Graphs} (DAG)~\cite{pegasus,deft1,grid_workflow_taxonomy}.
Conceptually, this level of abstraction of the workflow does not  involve issues of resource provisioning and utilization, monitoring, optimization, recovery from errors, as well as plethora of other items essential
for efficient execution of workflows in the distributed environment. These tasks are handled in the context of \textit{Workload Management} which we very briefly described in~\ref{from_grid_to_workload}.

In summary, \textbf{we make a distinction between the Workflow Management} domain which concerns itself with controlling the scientific workflow, \textbf{and Workload Management} which
is a domain of resource provisioning, allocation, execution control and monitoring of execution etc. The former is a level of abstraction above Workload Management, whereas the latter is in
turn a layer of abstractions above the distributed execution environment such as the Grid or Cloud.

% As we already noted, 
% Historically, in many cases the Workflow Management layer was not designed together with the underlying Workload Management system, and was (\textbf{FIXME})

\subsubsection{HTC vs HPC}
The term \textbf{High-Performance Computing} (HPC) is used in reference to systems of exceptionally high processing capacity (such as individual supercomputers, which are typically highly parallel systems),
in the sense that they handle substational workload and deliver results over a relatively short period of time. By contrast,\textbf{ High-Throughput Computing} (HTC) involves harnessing to a wider pool
of more conventional resources in order to deliver a considerable amount of computational power, although potentially over loger periods of time. 
\begin{quote}
\textit{``HPC brings enormous amounts of computing power to bear over relatively short periods of time. HTC employs large amounts of computing power for very lengthy periods.''}~\cite{htc}.
\end{quote}

In practice, the term \textit{HTC} does cover most cases of Grid Computing where remote resources are managed for the benefit of the end user and are often made available on prioritized and/or
opportunistic basis (e.g. the so-capped ``spare cycles'', utilizing temporary drops in resource utilization on a particular site to deploy additional workload thus increasing the overall system
throughput). Majority of the computational tasks of the LHC experiments were completed using standard off-the-shelf equipment rather than supercomputers. It is important to note, however,
that modern Workload Management Systems can be adapted to deliver payload to HPC systems such as Leadrship Class Facilities in the US, and such efforts are currently under way~\cite{panda_chep13}.

\subsubsection{The Role of Data Management}
In most cases of interest to us, data management plays a crucial role in reaching the scientific goals of an experiment. It is covered in detail separately (see Section~\ref{data}).
As noted above, it represents the \textit{data flow} component of the overall workflow management and therefore needs to be addressed here as well.

In a nutshell, we can distinguish between two different approaches to handling data -- on one hand, managed replication and transfers to sites, and on the other hand, network-centric
methods of access to data repositories such as \textbf{xrootd}~\cite{xrootd}.

An important design characteristic which varies widely among Workflow Management Systems is the degree of coupling to the Data Management components. That has significant impact
on reusability of these systems as a more tight coupling usually entails necessity of a larger and more complex stack of software than would otherwise be optimal and has other consequences.

\subsection{Examples}
\label{wms_examples}
\subsubsection{The Scope of this Section}
Workflow and Workload Management, especially taken in conjunction with Data Management (areas which they are typically interconnected) is a vast
subject and  covering features of each example of WMS in any detail would go well beyond the scope of this document. In the following, we provide references
to those systems which are more immediately relevant to HEP and related fields than others.

\subsubsection{HTCondor and glideinWMS}
\label{htcondor}
HTCondor~\cite{htcondor} is one of the best known and  important set of Grid and High-Throughput Computing (HTC) tools. It provides an array of functionality, such as
as a batch system solution for a computing cluster, remote submission to Grid sites (via Condor-G) and sutomated transer (stage-in and stage-out) of data.
In the past decade, HTCondor was augmented with a  Workload Management System layer, known as \textit{glideinWMS}~\cite{glideinwms}. The latter abstracts remote resources (Worker Nodes)
residing on the Grid and effectively creates a local (in terms of the interface) resource pool accessible by the user.
This is highly beneficial to the users already having familiarity with HTCondor since it shortens the learning curve.
On the other hand, deployment of this system is not always trivial and typically requires a central service to be operated with the desired degree of service level (the so-called ``glidein factory'').

HTCondor has other notable features. One of the most basic parts of its functionality is the ability to trasnfer data consumed and/or produced by the payload job according to certain rules set by the user. This works well
when used in local cluster situations and is somewhat less reliable when utilized at scale in the context of Grid environment. One of HTCondor components, \textit{DAGMan}, is a meta-scheduler
which uses DAGs (see~\ref{workflow_workload}) to manage workflows~\cite{dagman}.

\subsubsection{Workload Management for LHC Experiments}
This is the list of systems (with a few references to bibliography) utilized by the major LHC experiments - note that in each, we identify components representing layers or subdomains such as Workload Management etc:

\begin{center}
  \begin{tabular}{ c | c | c | c }
    \hline
    \textbf{Project} & \textbf{Workload Mgt} & \textbf{Workflow Mgt} & \textbf{Data Mgt}\\ \hline \hline
    ATLAS & PanDA~\cite{panda_chep10} & ProdSys2 & Rucio~\cite{rucio_chep13}\\ \hline
    CMS  & GlideinWMS~\cite{glideinwms} & Crab3~\cite{crab3_chep12} & PhEDEx~\cite{phedex_chep09,phedex_chep10}\\ \hline
    LHCb  & DIRAC~\cite{dirac_acat09}  & DIRAC Production Mgt & DIRAC DMS\\ \hline
    Alice  & gLite WMS~\cite{glite_chep09} & AliEn~\cite{alien_chep07} & AliEn~\cite{alien_chep07}\\ 
    \hline
  \end{tabular}
\end{center}

\subsubsection{@HOME}
\label{at_home}
There are outstanding examples of open source middleware system for volunteer and grid computing, such as BOINC~\cite{boinc} (the original platform behind SETI@HOME),
FOLDING@HOME and others. The central part of their design is the server-client architecture, where the clients can be running on a variety of platforms,
such as PCs and game consoles made available to specific projects by volunteering owners. Deployment on a cluster or a farm is also possible.

While this approach to distributed computing won't work well for most experiments of the LHC scale (where moving significant amounts of data presents a perennial problem)
it is clearly of interest to smaller scale projects with more modes I/O requirements. Distributed platforms in this class have been deployed, validated and used at scale.

\subsubsection{European Middleware Initiative}
The European Middleware Initiative~\cite{emi} is a concortium of Grid services providers (such as ARC, dCache, gLite, and UNICORE).
It plays an important role in the the Worldwide LHC Computing Grid (WLCG). The \textit{gLite}~\cite{glite_chep09} middleware toolkit was used by LHC experiments as one of methods
to achive resource federation on the European Grids. 

\subsection{Common Features}

\subsubsection{``Pilots''}
As we mentioned in ~\ref{from_grid_to_workload}, one of the primary functions of a WMS is to insulate the user from the heterogeneous and sometimes complex
Grid environment and certain failure modes inherent in it (e.g. misconfigured cites, transient connectivity problems, ``flaky'' worker nodes etc).
There is a proven solution to these issues, which involved the so called \textbf{late binding} approach  to the deployment of computational payload.

According to this concept, it is not the actual ``payload job'' that is initially dispatched to a  Worker Node residing in a
remote data center, but an intelligent ``wrapper'', sometimes termed a \textbf{``pilot job''}, which first validates the resource, its configuration and
some details of the environment (for example, outbound network connectivity may be tested). In this context, ``binding'' means matching process
whereby the payload job (such as a production job or a user analysis job) which is submitted to the WMS and is awaiting execution is assigned to a
live pilot which has already perfomed validation and configuration of the execution environment (for this reason, this technique is sometimes referred
to as \textbf{``just-in-time workload management''}). Where and how exactly this matching process happens is a subject of a design decision - in PanDA, it's done by
the central server, whereas in DIRAC this process takes place on the Worker Node by utilising the \textit{JobAgent}~\cite{dirac_chep10}.

With proper design,\textbf{ late binding} brings about the following benefits:
\begin{itemize}
\item The part of the overall resource pool that exhibit problems prior to actual job dispatch is excluded from the matching process by design.
This eliminates a very large fraction of potential failures that the user would otherwise have to deal with and account for, since the resource pool
exposed to the user (or for an automated client performing job submission) is effectively validated.

\item Some very useful diagnostic and logging capability may reside in the pilot. This is very important for troubleshooting and monitoring, which we shall discuss later.
Problematic resources can be identified and flagged at both site and worker node level.

\item In many cases, the overall latency of the system (in the sense of the time between the job submission by the user, and the start of actual execution) will be reduced --
due to the pilot waiting to accept a payload job -- leading to a more optimal user experience (agan, cf. the ``just-in-time'' reference).
\end{itemize}

DIRAC was one of the first systems where this concept was proposed and successfully implemented~\cite{dirac_acat09}. This approach also forms the architectural core
of the PanDA WMS~\cite{panda_chep12}.

In distributed systems where the resources are highly opportunistic and/or ephemeral, such as volunteer computing we mentioned in~\ref{at_home}, this variant
of the client-server model is the essential centerpiece of the design. In BOINC, the ``core client'' (similar to a ``pilot'') performs functions such as maintaining
communications with the server, downloading the payload applications, logging and others~\cite{boinc_client}.

In HTCondor and GlideinWMS (see~\ref{htcondor}) there is no concept of a sophisticated pilot job or core client, but there is a \textit{glidein} agent, which is deployed on Grid resources
and which is a wrapper for the HTCondor daemon process (\textit{startd}). One the latter is initiated
on the remote worker node it then the joins the HTCondor pool. At this point, matching of jobs submitted by the user to HTCondor \textit{slots} becomes possible \cite{glideinwms}.
While this ``lean client'' provides less benefits than more complex ``pilots'', it also belongs to the class of late-binding workload management systems, although at a simpler level.


\subsubsection{Monitoring}
In this section, we will need to refer to the concept of a \textbf{task}, and since this word is overloaded, we have to introduce a proper definition.
A few examples of WMS incorporate a concept of the task in the sense of a collection or grouping of jobs either created from the same template (and based on same executable)
and differentiated only by some input parameters or units of input data, or grouped together based on their interdependence and purpose (this is not a universal definition and it
does clash with ``tasks'' used in the batch system context but we'll stick with with in this chapter since it's widely used). The task in this case is a rudimentary case of workflow management tool.

All three domains involved in the present discussion (Workflow, Workload, Data) benefit from monitoring capability.

The ability of the user or operator of a WMS to get immediate and efficient access to information describing the status of jobs and tasks is an essential feature of a good Workload Management System~\cite{pandamon_chep10}.



\subsection{Opportunity for improvement}

Outline (TBD -mxp-)

Pilots, modularity, monitoring.



