\section{Survey of Current Landscape}

This section presents a general overview of the current landscape of HEP libraries and tools.  Each of the following sections provides a survey focused on one select \textit{area of opportunity} for improving on this landscape. 


\subsection{Overview}

\subsubsection{Editing notes}

\textit{(authors: Anders, Brett, all)}

\textit{This is not a real section and will be deleted}

\begin{itemize}
\item Give our understanding of the situation with libs/tools in HEP:
  \begin{itemize}
  \item General purpose libraries (I/O, statistical analysis, linear algebra)
  \item Data management and transfer tools
  \item Code management utilities
  \item High Throughput Computing (HTC) tools and services
  \item Graphics packages
  \item Documentation tools
  \item Build/release/scripting/testing tools
  \end{itemize}
\item describe overlap between libs/tools and the applications and systems groups
  \begin{itemize}
  \item attempt made to keep covered topics distinct from other two HEP-FCE groups
  \item work\textbf{flow} vs work\textbf{load} mgt
  \item geant4/ROOT as application or as lib/tool
  \end{itemize}
\item describe how libraries and tools tend to be reinvented for each major experiment 
\item describe the effort to adopt libs/tools from other experiments (Daya Bay: DBI from MINOS, SPADE from IceCube, Gaudi/GiGa/DetDesc from LHCb)
\item library-level parochialism (some possible topics):
  \begin{itemize}
  \item no time/effort/expertise to step back from immediate problem to look at bigger picture design (post-doc reads Geant4 manual, hacks up an application based on one of the examples and then stuff just starts to organically grow)
  \item single, monolithic geant4 application becomes center of analysis for whole experiment, everyone piles on
  \item inject ``just a few lines of code here'' instead of refactor
  \item configuration through code modification and recompilation
  \item cut-and-paste source code sharing instead of modules in shared libraries
  \item linkage of implementation code in shared libraries instead of proper pure-Interface libraries (C++)
  \item framework tendrils work their way throughout all code
  \item no framework (or even with a framework) leads to data-file-as-module-interface (I/O inefficient, scales poorly to large processing or complex workflows)
  \item heavy framework enforcing execution model (eg, I want to be
    able to test my code via interactive \texttt{ipython} prompt
    \textbf{and} have it run from a framework ``algorithm/module'')
  \end{itemize}
\item Social/psychological - maybe belongs in its own ``area of opportunity''?
  \begin{itemize}
\item collaborations lack support/time, or feel like the do, to write or adopt general purpose libs/tools and then support them going forward.
\item the community as a whole lacks common communication tools, rather individual projects/experiments all have their own way of doing things, or not at all.  Some tools are counter productive (eg, choosing web fora over email lists for conversations)
\item Communities that attempt to establish are often not responsive enough to users to maintain themselves.  Strive for a community healthy enough to have developers supporting experts, experts supporting users and users supporting newbies.
\item reward desired behavior: be gracious with bug reports, welcome and record feature requests from users and sometimes even satisfy them, thank and credit community members who contribute, do this in a non-BS, non-PR way (no auto email replies!)
\item Expertise in writing good, general-purpose libs/tools does not always imply expertise in promoting/advertising the software.  Need ways to uncover hidden diamonds.
\item An experiment's software is no better than the best software expert involved.
\item Loudest is not always best.
\item Large software decisions are often made without critical opposition, without a decision making policy in place, or ignored if one exists, and without full information readily available to form an objective selection.
\item Politics and familiarity often trump rationality and quality in selecting software or software strategies.
\item Prototypes become too loved to be replaced.
\item Proper software design and experience is looked down on as a hindrance by those who work closer to the Physics.
\item Software developers often design to the lowest user instead of expecting the users to be able to lift themselves up to the higher design.
\item Many decisions are made based on what may happen not what is likely to happen.
  \end{itemize}

\item Intro to remaining specific problems:
\begin{itemize}
\item the specific areas of opportunity identified as important where improvement (cross-cutting) has potential to help many.
\item First two have a relatively large phase space with many sub components, second two are somewhat smaller in scope.
\end{itemize}

\end{itemize}
\textit{(end of notes)  }

\subsubsection{Areas of Opportunity}

Each of the following sections focus on one particular \textit{area of opportunity} to make improvements in how the community shares libs/tools between experiments.  In each area of opportunity we present:

\begin{itemize}
\item A description of the area.
\item Our understanding of what works and does not work to make the libs/tools in the area useful in a general, cross-experiment way.
\item Examples of current or past libs/tools to emulate or to serve as anti-patterns to avoid.
\item Specific aspects that need improvement and an estimation of what efforts would be needed to obtain that.
\end{itemize}
