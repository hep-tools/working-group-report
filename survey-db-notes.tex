\subsubsection{Editing notes}


\fixme{This is not a real section and will be deleted}


\begin{itemize}
\item Tech arguments, make some collection of unbiased facts to judge suitability (is it possible?)

	Essentially all experiments will use one or more databases to
	store their data (or pointers to the data), as well as provenance 
	information and metadata.  The requirements for these databases will
	vary widely across the range of possible experiments, as will
	requirements for access (read and write) to different databases.  
	We focus here on the possibility of each experiment presenting
	to its collaboration a `Conditions Database' interface, which
	provides a common (primarily read) access interface to universal
	information that is widely needed by the collaboration.

	Examples of this information might be:
	\begin{itemize}
	\item database table list, DB schema
	\item event catalog parameter description, schema
	\item high level experiment run catalog
	\item high level experiment event catalog
	\item high level data processing runs catalog
	\end{itemize}

	One can imaging that query access to such common information be
	available in a uniform fashion to facilitate, for instance,
	automated lookup of DB schema.  

\item Focus on conditions database as one particularly cross-experiment component and make this clear to avoid confusion.
  \begin{itemize}
  \item MINOS's ``DBI'' adopted by other neutrino experiments
  \item BaBar/Atlas/LHCb's similar
  \item DB schema, fixed parts + experiment-specific parts
  \end{itemize}
\item Note that configuration DB (eg. for configuring the experiment DAQ) may be too experiment-specific to provide one solution
  \begin{itemize}
  \item however, system to sink+emit alarms/messages from disparate providers to disparate consumers would be generally useful (``XMLBlaster'' is one example toolkit)
  \end{itemize}
\end{itemize}
