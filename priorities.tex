\section{Prioritized Recommendations}

\subsection{Cross-Experiment Effort}

This section presents priority recommendations that the working group
gives to the HEP-FCE with an understanding that they are of interest
to funding entities.

\begin{enumerate}
\item The goals of the HEP Software Foundation are
  largely overlapping with HEP-FCE.  The two groups can be mutually
  beneficial.  HSF is forming a general, grassroots organization while
  HEP-FCE potentially can fund specifically targeted efforts.  HSF can
  be a source of requirements, working groups, review panels, project
  proposals and expertise all of which HEP-FCE can beneficially
  leverage to carry out its goals.  It is recommended that HEP-FCE
  continue to participate in HSF formation activities and look into
  how the two groups might partner on specific actions.

\item Various detailed ``opportunities'' listed in the following
  survey sections call out the need for further work to be carried out
  in some detail by technical working groups. These are needed to
  better understand the nature of a specific problem shared across
  many experiments, formulate requirements for and in some cases
  design and implement solutions.  The HEP-FCE should organize such
  working groups from suitable expertise from the HEP software community.

\item Packages or frameworks (or significant subsets) which have proven 
popular (used by more than one experiment) and useful should be considered for
cross-experiment support, especially in terms of providing support for
easy adopability (setup and install by other experiments, on other 
O/S platforms) and documentation (detailed guides and non-experiment
specific manuals).

documentation

\end{enumerate}


\subsection{Effort by Experiments}

Throughout the following survey sections, a number of best practices
and pitfalls relevant to the development and use of software libraries
and tools by individual experiments were identified.

\begin{enumerate}

\item New experiments should not underestimate the importance of
  software to their success.  It should be treated as a major
  subsystem at least on par with other important aspects such as
  detector design, DAQ/electronics, civil construction, etc.

\item Experiments should understand the pitfalls listed in
  section~\ref{subsec:pitfalls}).  New experiments should plan and
  implement mechanisms to avoid them and existing experiments should
  reflect on which ones may apply and develop ways to address them.
  Likewise, the best practices listed in
  section~\ref{subsec:bestpractices}) should be considered.  New
  experiments should attempt to follow them and if practical and
  beneficial, existing experiments should seek to make the changes
  needed to implement them.

\item New and existing experiments should join the effort to organize
  and otherwise supply representative members to the HEP Software
  Foundation.

\item Aspects of successful Event Processing Software Frameworks include: 
those with flexible (possibly hierarchical) notions of ``events'', 
those that are easily adopable by new experiments, are well-documented, 
have dynamically configuable (possibly scriptable) configuration 
parameter sets and are modular and efficient (allow C++ like modules 
for low level 
operations combined with scripting layer like python for flexible 
higher level control.

\item Aspects of successful Software Development tools include:
those that follow licence-free availablity and free-software distribution models, those that include code repositories, build systems that work on a variety of platforms with a small number of clearly defined base element dependencies 
(i.e. C compiler, compression library, specific version of python), release 
configuration systems with versioning that understand a variety of 
platforms; those that support automatic continuous integration and 
regression testing; those that have open documentation updating and  
bug-reporting and tracking.

\item Aspects of successful Data Management tools include:
those that are content neutral; those that are properly layered depending on the scale (small or large) of the experiment; those that properly and 
completely integrate metadata including provenance information;  those that
can be developed locally and grown organically.

\item Aspects of successful Workflow and Workload Management tools include:
those that understand the distinction between and support flexible, efficient 
interaction between workflow, workload, and data management aspects 
of a system; those that make efficient use of resources (CPU, RAM-memory, 
disk, network, tape) for processing in parallel;  those that allow 
granular, multi-level monitoring of status; those that handle error 
cases effectively and inclusively; those that are properly scaled 
to the size of the experiment.

\item Aspects of successful Geometry Information Management tools include:
those that follow or set widely used standards for representation of
geometric information; those that follow standards for visualization.

\item Aspects of successful Conditions Database tools include:
those that allow standardized, experiment-wide access to representative 
or specific event conditions so that realistic simulations or statistics
can be generated by users without detailed knowledge of detectors or specific
event.

\end{enumerate}

