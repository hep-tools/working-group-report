\section{Data Management Editing notes}

\subsubsection{A few notes by Maxim}

Dear All,

I created this file because I wanted to write down a few thoughts but didn't want to obfuscate
the latex source for this section or create a conflict at check-in.

Maxim -mxp-

\begin{itemize}

\item Do we need to define file catalog as something complementary to ``true'' metadata? In the sense that it's a translation to path/localization but doe not necessarily carry other info. [awb] But it may contain a lot of metadata. For example, the Fermi Space Telescope Data Catalog contains a large set of meta-data which enables users to do queries. In that sense, the Data Catalog is (just) a meta-database i.e. it's a convenient way to organize meta-data.

\item Evolution of Metadata - do we need to mention event indexing?

\item File catalog - right now there is always a centralized DB which may become a bottleneck in some instances.
Is it possible or desirable to have a truly distributed DB for better availability and speed? Consistency is the main problem

\item Coordinated push to sites vs on-demand access via network (cf. xrootd) - to which extent can the latter dominate technology choices?

\item Do we need to take a look at Globus Online and its possible limitations with regard to policy/pricing/scalability?

\item I already covered the mutual dependency of three domains in my WMS section:
\begin{itemize}
\item Workflow mgt
\item Workload mgt
\item Data mgt
 \end{itemize}
 There needs to be a tie-in in this section as well. Data management is congruent with workflow (data component) but is ultimately
 translated to workload and what the latter does with the data.
 
 \item Most systems deployed at scale in HEP are more or less experiment specific.
 
 \item Network performance monitoring and its role in data distribution.
 
 \end{itemize}

