\section{Event Processing Software Frameworks}
\textit{(authors: Brett, ??)}

\subsection{Description}

A software framework abstracts common functionality expected in some
domain.  It provides some generic implementation of a full system in
an abstract way that lets application-specific functionality to be
added through a modular implementation of framework interfaces.

Whereas toolkit libraries encourages different functionality through
developing different applications a framework provides the overall
processing control function with typically one main entry point.
Different functionality is obtained by configuring a the framework to
use different module implementations.

In the context of HEP software, the term ``event'' is probably the
most overloaded and poorly defined term that is commonly used.  The
term ``module'' is probably second.  Both are used here.  In this
context, an ``event'' is a unit of data of a scope which is dependent
on the ``module'' of code which is processing.  In the context of a
code module that generates initial kinematics an event is the
information about the interaction.  In a module that simulates the
passage of particles through a detector an event may contain all
energy depositions in active volumes.  In a detector electronics
simulation it may contain all signals collected from these active
volumes.  In a trigger simulation module, it would be all readouts of
these signals above some threshold or other criteria.  At this point
data from real detectors gain symmetry with simulation.  Beyond this
data reduction, calibration, reconstruction and other analysis modules
each have a unique concept of the ``event'' of data they operate on.
Depending on the nature of the physics, detector, and follow on
analysis, every module may not preserve the multiplicity of data.  For
example, a single interaction may produce multiple triggers, or none.

With that description, an event processing software framework is
largely responsible to marshal data through a series (in general a
graph) of such code modules which then mutate the data.  To support
these modules the framework provides access to external services such
as those that give access to the data itself, handle file I/O, access
to descriptions of the detectors, provide for visualization or
statistical summaries, and databases of conditions for applying
calibrations.

The defining nature of these frameworks are to aggregate and organize
code modules provided by a disparate set of developers.  As such,
quality frameworks provide means to manage this complexity.  The
modular design itself is one.  Particularly in C++ based frameworks,
these modules are implemented by inheriting from a framework-provided
base class. In order to avoid hard-linking which modules to use at
compile time they are best put into shared libraries which can be
loaded at run time as driven by a user configuration layer provided by
the framework.  Even with this, compile-time dependencies can balloon
build times and frameworks should best implement the base classes with
pure-virtual interface classes such that changes to one part of the
code base do not require a full rebuild/relink.

\subsection{Gaudi}

The Gaudi event processing framework provides a comprehensive set of
features and is extensible enough that it is suitable for a wide
variety of experiments.  It was conceived by LHCb and adopted by ATLAS
and these two experiments still drive its development.  It has been
adopted by a diverse set of experiments including HARP, Fermi/GLAST,
MINER$\nu$A, Daya Bay and others.  The experience of Daya Bay is
illuminating for both Gaudi specifically and of more general issues of
this report.

First, this adoption was greatly helped by the support from the LHCb
and ATLAS Gaudi developers.  Although not strictly their
responsibility, they found the time to offer help and support to other
experiments.  Without this, the success of the adoption is uncertain
and at best would have taken much more effort.  Daya Bay recognized
the need and importance of such support and, partly selfishly, formed
a mailing list\cite{gauditalk} and got Gaudi developers from many
experiments involved.  It became a forum the more efficiently spread
beneficial information from the main developers as well as offloaded
some support to new experts from the other experiments to help
themselves.

Other aspects were beneficial to adoption.  The Gaudi build system
based on CMT was cross platform, open and easy to port.  It had layers
of functionality (package build system, release build system, support
for experiment packages and ``external'' ones) but it did not require
all-or-nothing adoption.  This allowed for some staged approach that
allowed Daya Bay to get started using the framework more quickly.

The importance of having all Gaudi source code open and available can
not be diminished.  Also important was Gaudi developers inclusion of
the growing community in the release process.

There were, however areas that would improve the adoption of Gaudi.
The primary one would be direct guides on how to actually adopt it.
This is something that must come from the community and likely in
conjunction with some future adoption.  Documentation on Gaudi itself
was also a problem particularly for Daya Bay where many of the basic
underlying framework concepts where new.  Older design documents and
some experiment-specific ones were available but not always accurate
nor focused on what was needed.  Over time Daya Bay produced it's own
Daya Bay-specific documentation which unfortunately perpetuates this
problem.

While Gaudi's CMT-based package and release build system ultimately
proved very useful, it hampered initial adoption as it is not commonly
used and required understanding.  It is understood that there is a
movement to provide a CMake based build system which may alleviate
this hurdle.

Finally, although Gaudi is full featured and flexible it does not come
with all framework level functionality and, in its core, misses some
important extensions.  In particular, Daya Bay adopted three Gaudi
extensions from LHCb's code base.  These are actually very general
purpose but due to historical reasons were not provided separately.
These were GaudiObjDesc (data model definition), GiGa (Geant4
interface) and DetDesc (detector description).  Some extensions
developed by other experiments were rejected and in-house
implementations were developed.  In particular the extension that
provided for file I/O was considered too much effort to adopt.   

One aspect of Gaudi, as a framework, that left it unsuitable for Daya
Bay had to do with the event processing model.  Unlike collider
experiments, Daya Bay necessarily had to deal with a non-sequential,
linear event stream.  Multiple detectors at multiple sites produced
data in time order but not synchronously.  Simulation and processing
did not preserve the same ``event'' multiplicity.  Multiple sources of
events (many independent backgrounds in addition to signal) must be
properly mixed.  Finally, of the strongest background rejection
criteria to select inverse beta decay from anti-neutrino interactions
is to require a coincidence in time between a prompt positron
interaction in the scintilator followed by delayed activity from the
absorption of a neutron.  The flexibility of Gaudi allowed Daya Bay to
extend its very event processing model to add the support necessary
for these features.  



\subsection{What works, what doesn't}
\subsection{Examples}
\subsection{Opportunity for improvement}




\subsubsection{Editing notes}



\textit{This is not a real section and will be deleted}

\begin{itemize}
\item types of software frameworks
\item focus on event data processing FW:
  \begin{itemize}
  \item provides a ``main'' program
  \item determines execution model
  \item provides hooks/entries for user/programmers to supply
    \begin{itemize}
    \item start/end of run/subrun/file
    \item once-per-event entry
    \end{itemize}
  \item end-user job configuration and scripting
  \item provides functionality to manage compile- and run-time complexity, (Interface classes, shared library autoload/plugin)
  \item provides software development and release management support
  \end{itemize}
\end{itemize}
Some frameworks
\begin{itemize}
\item Gaudi
\item RAT
\item IceTray
\item Art
\end{itemize}
Comparisons:
\begin{itemize}
\item for Daya Bay \url{https://wiki.bnl.gov/dusel/index.php/Software_Framework_Comparison}
\end{itemize}
